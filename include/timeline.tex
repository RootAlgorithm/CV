%-----------------------------------------------------------------------------------------------------------------------------------------------%
%	The MIT License (MIT)
%
%	Copyright (c) 2016 Jan Küster
%
%	Permission is hereby granted, free of charge, to any person obtaining a copy
%	of this software and associated documentation files (the "Software"), to deal
%	in the Software without restriction, including without limitation the rights
%	to use, copy, modify, merge, publish, distribute, sublicense, and/or sell
%	copies of the Software, and to permit persons to whom the Software is
%	furnished to do so, subject to the following conditions:
%	
%	THE SOFTWARE IS PROVIDED "AS IS", WITHOUT WARRANTY OF ANY KIND, EXPRESS OR
%	IMPLIED, INCLUDING BUT NOT LIMITED TO THE WARRANTIES OF MERCHANTABILITY,
%	FITNESS FOR A PARTICULAR PURPOSE AND NONINFRINGEMENT. IN NO EVENT SHALL THE
%	AUTHORS OR COPYRIGHT HOLDERS BE LIABLE FOR ANY CLAIM, DAMAGES OR OTHER
%	LIABILITY, WHETHER IN AN ACTION OF CONTRACT, TORT OR OTHERWISE, ARISING FROM,
%	OUT OF OR IN CONNECTION WITH THE SOFTWARE OR THE USE OR OTHER DEALINGS IN
%	THE SOFTWARE.
%
%-----------------------------------------------------------------------------------------------------------------------------------------------%

% define global counters
\newcounter{yearcount}


\newcounter{leftcount}

% env cvtimeline
%
% creates a vertical cv timeline
%
% param 1: start year
% param 2: end year
% param 3: overall width
% param 4: overall height
\newenvironment{cvtimeline}[4]{
	
	\newcommand{\cvcategory}[2]{
		\node[label=\mbox{\colorbox{##1}{\strut\hspace{2pt}}\colorbox{white}{\textcolor{text}{##2}}}] at (0,-5) {}; %start year
	}

	\newcommand{\bxwidth}{4.5}
	\newcommand{\bxheight}{2}


	% creates a stretched box as cv entry headline followed by two paragraphs about 
	% the work you did
	% param 1:	event start month/year
	% param 2:	event end month/year
	% param 3:  event name
	% param 4:	institution (where did you work / study)
	% param 5:	what was your position
	% param 6:	color
	% param 7: level (position, use minus for left placement)
	\newcommand{\cvevent}[7] {
		
		\foreach \monthf/\yearf in {##1} {
			\foreach \montht/\yeart in {##2} {

				\pgfmathparse{#3/\fullrange*((\yearf-#1)+(\monthf/12))}
  				\let\startexp\pgfmathresult
				\pgfmathparse{#3/\fullrange*((\yeart-#1)+(\montht/12))}
  				\let\endexp\pgfmathresult
				\pgfmathparse{1/(\endexp-\startexp+1)}
  				\let\lenexp\pgfmathresult
				\pgfmathparse{0.5*\endexp+0.5*\startexp}
  				\let\midexp\pgfmathresult

				\filldraw[fill=##6,draw=none, opacity=0.9] (-0.15-##7,\startexp) rectangle (-0.25-##7,\endexp);
				\draw[draw=##6, line width=1.5pt] (0.05, \startexp) -- (1,\startexp);

                \node[label={[label distance=0]0:\scriptsize\colorbox{##6}{\strut}\colorbox{white}{\textcolor{mgray}{##1}\hspace{3pt}\textcolor{text}{##3}}}] at (0.5, \startexp) {};
                \node[label={[label distance=0]0:\tiny\colorbox{white}{\textcolor{text}{##4}}}] at (0.75, \startexp -0.35) {};
			}
			\addtocounter{leftcount}{1}
		}
	}

	%--------------------------------------------------------------------------------------
	%	BEGIN
	%--------------------------------------------------------------------------------------

	\begin{tikzpicture}

	\setcounter{leftcount}{1}

	%calc fullrange= number of years
 	\pgfmathparse{(#2-#1)}
  	\let\fullrange\pgfmathresult
	\draw[draw=text,line width=3pt] (0,0) -- (0,#3) ;	%the timeline

	%for each year put a horizontal line in place
	\setcounter{yearcount}{1}
	\whiledo{\value{yearcount} < \fullrange}{
		\draw[fill=white,draw=text, line width=2pt]  (0,#3/\fullrange*\value{yearcount}) circle (0.1);
		\stepcounter{yearcount}
	}

	%start year
	\filldraw[fill=white!100,draw=text,line width=3pt] (0,-0.5) circle (0.5);
	\node[label=\textcolor{text}{\textbf{\small#1}}] at (0,-0.85) {}; 

	%end year
	\filldraw[fill=white!100,draw=text,line width=5pt] (0,#3+0.75) circle (0.65);
	\node[label=\textcolor{text}{\textbf{\large#2}}] at (0,#3+0.42) {}; 


	
}%end begin part of newenv
{\end{tikzpicture}}